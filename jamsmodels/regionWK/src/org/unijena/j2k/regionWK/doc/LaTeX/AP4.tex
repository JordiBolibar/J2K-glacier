\section{Methoden zur Bestimmung der Niederschlagsform}
\subsection{Bestimmung von Niederschlagsform sowie Schnee- und Regenanteilen}
\subsubsection*{Sourcen:} 	org.unijena.regionWK.AP4.CalcRainSnowParts.java
\subsubsection*{Beschreibung:}
In Abh�ngigkeit von der Temperatur wird bestimmt, ob es sich beim gemessenen Niederschlag um Schnee oder Regen handelt. Dabei wird angenommen, dass es einen Temperatur-�bergangsbereich gibt, in dem sowohl Regen als auch Schnee bzw. Mischniederschl�ge auftreten k�nnen. Es muss ein Temperaturwert ($Trs$ in �C) angegeben werden, der der Temperatur entspricht, bei der 50\% des Niederschlages als Schnee und 50\% als Regen fallen. Zus�tzlich muss ein Parameter $Trans$ (in K) bestimmt werden, der der halben Breite des �bergangsbereiches entspricht. Der tats�chliche Schneeanteil ($p(s)$) am Tagesniederschlag in Abh�ngigkeit von der Lufttemperatur ($T$) berechnet sich dabei nach: 

\begin{equation}
p(s) = \frac{TRS + Trans -T}{2 \cdot Trans} \quad \mathrm{[mm]}
\end{equation}

Die t�gliche Schneemenge ($N_S$) bzw. Regenmenge ($N_R$) des gesamten Niederschlags ($N$) ergibt sich nach:

\begin{equation}
N_S = N \cdot p(s) \quad \mathrm{[mm]}
\end{equation}

\begin{equation}
N_R = N \cdot (1-p(s)) \quad \mathrm{[mm]}
\end{equation}

\renewcommand{\refname}{\subsubsection*{Literatur}}
\begin{thebibliography}{}
\bibitem[\textsc{JAMSWiki} (2008)\textsc{JAMSWiki} ]{JAMSWiki}
\textsc{JAMSWiki} (2008): Hydrologisches Modell J2000, \\ $\left\langle http://jams.uni-jena.de/jamswiki/index.php/Hydrologisches\_Modell\_J2000 \right\rangle$ (Stand: 18.2.2008) (Zugriff: 15.12.2008).
\end{thebibliography} 

\subsection*{Eingabe:} 

\begin{tabular}{lllp{7.5cm}ll}
%& & &\\ 
& $area$ & & Gr��e des betrachteten Gebiets & $\mathrm{[m^{2}]}$ \\
& $snow\_trs$ & & Temperaturwert bei dem 50\,\% Regen und 50\,\% Schnee fallen& $\mathrm{[�C]}$\\
& $snow\_trans$ & & Halbe Breite des �bergangsbereiches & $\mathrm{[K]}$\\
& $tmean$ & & Mittlere Temperatur & $\mathrm{[�C]}$\\
& $precip$ & & Niederschlag & $\mathrm{[mm]}$\\
\end{tabular}\\ 

\subsection*{Interne Variablen:}

\begin{tabular}{lllp{8.5cm}ll}
%& & &\\ 
& $pSnow$ &  & relativer Schneeanteil in Abh�ngigkeit von der Temperatur& $\mathrm{[-]}$\\
\end{tabular}\\

\subsection*{Ausgabe:}
\begin{tabular}{lllll}
& $rain$ & & Regen & $\mathrm{[mm]}$\\
& $snow$ & & Schnee & $\mathrm{[mm]}$\\
& $mixPrecip$ & & Mischniederschlag $\mathrm{(ja | nein)}$ & $\mathrm{[true | false]}$\\
\end{tabular}\\
   
  

